\begin{frame}
	\frametitle{Introduction}
	The ideal gas equations form a set of hyperbolic differential equations of the form
%	
	\begin{align*}
		\frac{\partial \mathbf{U}}{\partial t} + \Delta \cdot \mathbf{F} = 0
	\end{align*}
%	
	In this project, consider the 1D advection of the mass density $\rho$ with a global constant velocity $u$:
%	
	\begin{align*}
		\frac{\partial \rho}{\partial t} + u \frac{\partial \rho}{\partial x}  = 0
	\end{align*}
%	
	The analytical solution is given by any function $q(x)$ with $\rho(x, t) = q(x-ut)$, which is $q(x)$ again translated by $ut$ along the $x$-direction.
\end{frame}



\begin{frame}
	\frametitle{Solution Scheme: Finite Volume Method}
%	\begin{block}{Finite difference methods}
%		Approximate differential operators through finite difference approximations
%	\end{block}
%	\begin{block}{Finite volume methods}
%		Make use of the divergence theorem: Convert volume integrals of finite volumes (cells) of divergence terms to surface integrals of fluxes though surfaces.
%	\end{block}	
%	\begin{block}{Method of Lines}
%		Approximate all but one derivative by finite differences. The equation for
%	\end{block}	
%	\begin{block}{Spectral Methods}
%		Represent solution by a linear combination of functions, transforming the PDE to algebraic equations or ODEs, e.g. by applying Fourier techniques.
%	\end{block}	

	Divide space into $N$ cells of same size.

	Make use of the divergence theorem: Convert volume integrals of finite volumes (cells) of divergence terms to surface integrals of fluxes though surfaces.
	
	Advantage for conservation laws: Finite volume method is conservative, because $F_{i+\tfrac{1}{2}} = F_{(i+1)-\tfrac{1}{2}}$
\end{frame}






\begin{frame}
	\frametitle{Solution Scheme: Problems}
	
	\begin{block}{Direction of information}
		In this problem, all characteristics propagate downstream: Information strictly travels in the flow direction.\\
		$\Rightarrow$ $\rho_i$ mustn't depend on downstream value. Otherwise, you'll be using ``information that isn't there yet''.\\
		$\Rightarrow$ The discretisation will depend on the sign of the velocity $u$.
	\end{block}

	\begin{block}{CFL Condition}
		For explicit integration methods, we need to limit the time step such that information can travel at most by 1 cell during each time step to avoid instabilities.\\
		In 1D: $\Delta t \leq \Delta x / u\quad$ ;  In $N$D: $\Delta t \leq [\sum_i^N u_i / \Delta x_i ]^{-1}$
	\end{block}

	\begin{block}{Numerical Diffusion}
		The solution is not advected perfectly (except for $\Delta t = \frac{\Delta x}{u}$) but smoothed out. The numerical algorithm has a diffusion term as a by-product For a first order algorithm, we're effectively solving $\frac{\partial \rho}{\partial t} + u \frac{\partial \rho}{\partial x} = \frac{\Delta x u}{2}\frac{\partial^2 \rho}{\partial x^2}$
	\end{block}

\end{frame}






\begin{frame}
	\frametitle{Periodic Boundary Conditions}
	
	Simulate a periodic boundary by introducing ghost cells:



	\begin{tikzpicture}[xscale=.9,yscale=1,samples=400, transform shape,every node/.style={scale=.8}]
		%===============
		%	Sequential
		%===============
		
		
		%real Grid
		\filldraw[fill=green!10,draw=teal!80, line width=.15mm] (0, 0) rectangle (8, 1);
		\draw[step=1cm,lightgray,very thin] (0, 0) grid (8,1);
		\draw[draw=teal!80, line width=.15mm] (0, 0) rectangle (8, 1);

		% pattern for real cells
		\fill[pattern=horizontal lines, pattern color=teal!40]  (0,0) rectangle (2, 1);	
		\fill[pattern=horizontal lines, pattern color=teal!40]  (6,0) rectangle (8, 1);	
			
		% Ghost cells left	
		\filldraw[fill=red!20,draw=red!80, line width=.15mm] (-2, 0) rectangle (0, 1);
		\draw[step=1cm,lightgray,very thin] (-2, 0) grid (0,1);
		\draw[draw=red!80, line width=.15mm] (-2, 0) rectangle (0, 1);
		
		%Ghost cells right
		\filldraw[fill=red!20,draw=red!80, line width=.15mm] (8, 0) rectangle (10, 1);
		\draw[step=1cm,lightgray,very thin] (8, 0) grid (10,1);
		\draw[draw=red!80, line width=.15mm] (8, 0) rectangle (10, 1);	


		% Arrows
		\draw[<-, black!70, thick] (-1,1.2) to[bend right=-10] (7,1.2);
		\draw[<-, black!70, thick] (9,-0.2) to[bend right=-10] (1, -0.2);
	
	\end{tikzpicture}
\end{frame}
