\documentclass[10pt]{beamer}


\usepackage{lmodern} 		% Diese beiden packages sorgen für echte 
\usepackage[T1]{fontenc}	% Umlaute.

\usepackage{amssymb, amsmath, color, graphicx, float, setspace, tipa}
\usepackage[utf8]{inputenc} 
\usepackage[english]{babel}
\usepackage[justification=centering]{caption}
\addto\captionsenglish{\renewcommand{\figurename}{}} %Abbildungen nicht bzw. anders beschriften.


%\usepackage[pdfpagelabels,pdfstartview = FitH,bookmarksopen = true,bookmarksnumbered = true,linkcolor = black,plainpages = false,hypertexnames = false,citecolor = black, breaklinks]{hyperref}
%\usepackage{url}
%\usepackage{picins} 		%Gleittext um Grafik. Befehl: parpic. Vorlage siehe unten
\usepackage{longtable} 		%Seitenübergreifende Tabelle. Vorlage siehe unten
\newtheorem*{bem}{Bemerkung} % Neue Theorem-Umgebung: Bemerkung
\newcommand{\fillframe}{\vskip0pt plus 1filll} 
\newcommand{\musr}{$\mu$SR }

\usepackage{units}
\newcommand{\half}{\nicefrac{1}{2}}

\usepackage{tikz}
\usetikzlibrary{patterns}

\usepackage{grffile} %allow image filenames.that.include.dots.png



%-----------------
%BEAMER-SPEZIFISCH
%-----------------


\usetheme{metropolis}
%deactivate a new page when a new section begins
\metroset{sectionpage=none} 
\usepackage{FiraSans}
\usefonttheme[onlymath]{serif}



% Verschiedene Varianten von usetheme, usecolortheme und usefonttheme kann man hier ausprobieren: http://deic.uab.es/~iblanes/beamer_gallery/


%Halbtransparente Overlays (was als nächstes Element auf der Folie gezeigt wird)
%\setbeamercovered{transparent} 

% Entfernt Navigationssymbole unten
%\beamertemplatenavigationsymbolsempty 

% Seitenzahlen als links
%\setbeamertemplate{footline}[frame]  
%    \setbeamertemplate{footline}{%
%    	\raisebox{5pt}{\makebox[\paperwidth]{\hfill\makebox[10pt]{\hyperlink{tableofcontents}{\scriptsize\insertframenumber}}}}}






%---------------------
%--Metainformationen--
%---------------------
\title{Advection Project}


\author[M. Ivkovic]{Mladen Ivkovic}

\date{January 2018} 



% \title[Kurzform]{Vortrag zur Berechenbarkeit}
%     Titel des Vortrages
% \subtitle[Kurzform]{Untertitel}
%     Untertitel
% \author[M. Schulz]{Michael Schulz}
%     Autor festlegen
% \institute[IfI -- HU Berlin]{Institut für Informatik\\ Humboldt-Universität zu Berlin}
%     Angabe des Institutes
% \date[26.05.06]{26. Mai 2006}
%     Datum der Präsentation, alternativ kann mittels \date{\today} auch das aktuelle Datum eingetragen werden.
% \logo{\pgfimage[width=2cm,height=2cm]{hulogo}}
%     Die Datei hulogo.pdf (bzw. hulogo.png, hulogo.jpg, hulogo.mps bei Verwendung von pdftex als Backend) als Logo auf allen Folien, hier mithilfe des Paketes pgf.
% \titlegraphic{\includegraphics[width=2cm,height=2cm]{hulogo}}
%     Die Datei hulogo.pdf (bzw. analog wie bei \logo auch entsprechendes Format) als Logo nur auf der Titelseite unter Verwendung des Paketes graphicx.








%===================================================================================
%===================================================================================
% \begin{frame}[Overlay-Aktionen][Optionen]{Titel}{Untertitel}
% 
% Overlay-Aktionen
%     Overlay-Aktionen setzen die Standard-Overlay-Aktionen aller Umgebungen innerhalb des Frames, welche Aktion-Spezifikationen erlauben. Dazu gehören u.a. \item bei Listen und Block-Umgebungen.
% 
%     <+->
%         Sorgt dafür, dass die Elemente stückweise zum Vorschein kommen.
% 
% Optionen
% 
%     allowdisplaybreaks
%         Sorgt durch Aufruf von \allowdisplaybreaks aus AMS-LaTeX für einen Seitenumbruch bei mehrzeiligen Formelumgebungen. Funktioniert nur im Zusammenhang mit der Option allowframebreaks
%     allowframebreaks
%         Passt der Inhalt nicht mehr auf ein Slide, wird er automatisch auf mehrere Slides verteilt. Allerdings ist somit kein Overlay mehr möglich.
%     b,c,t
%         Sorgt dafür, dass der Frame nach unten (b), zentriert (c) oder nach oben (t) ausgerichtet wird.
%     fragile
%         Wird für Quelltextumgebungen, z.B. verbatim, benötigt.
%     label=name
%         Legt einen Namen für ein Frame fest um es später mit \againframe{name} erneut aufrufen zu können.
%     plain
%         Unterdrückt die Anzeige der Überschrift, Fußzeile und Sidebar.
%     squeeze
%         Verkleinert die vertikalen Abstände so weit wie möglich um u.U. mehr auf der Folie unterbringen zu können.
% 
%===================================================================================
%===================================================================================




\begin{document}


\begin{frame}{}
	\titlepage
\end{frame}




\section{Introduction}
\begin{frame}
	\frametitle{Introduction}
	The ideal gas equations form a set of hyperbolic differential equations of the form
%	
	\begin{align*}
		\frac{\partial \mathbf{U}}{\partial t} + \Delta \cdot \mathbf{F} = 0
	\end{align*}
%	
	In this project, consider the 1D advection of the mass density $\rho$ with a global constant velocity $u$:
%	
	\begin{align*}
		\frac{\partial \rho}{\partial t} + u \frac{\partial \rho}{\partial x}  = 0
	\end{align*}
%	
	The analytical solution is given by any function $q(x)$ with $\rho(x, t) = q(x-ut)$, which is $q(x)$ again translated by $ut$ along the $x$-direction.
\end{frame}



\begin{frame}
	\frametitle{Solution Scheme: Finite Volume Method}
%	\begin{block}{Finite difference methods}
%		Approximate differential operators through finite difference approximations
%	\end{block}
%	\begin{block}{Finite volume methods}
%		Make use of the divergence theorem: Convert volume integrals of finite volumes (cells) of divergence terms to surface integrals of fluxes though surfaces.
%	\end{block}	
%	\begin{block}{Method of Lines}
%		Approximate all but one derivative by finite differences. The equation for
%	\end{block}	
%	\begin{block}{Spectral Methods}
%		Represent solution by a linear combination of functions, transforming the PDE to algebraic equations or ODEs, e.g. by applying Fourier techniques.
%	\end{block}	

	Divide space into $N$ cells of same size.

	Make use of the divergence theorem: Convert volume integrals of finite volumes (cells) of divergence terms to surface integrals of fluxes though surfaces.
	
	Advantage for conservation laws: Finite volume method is conservative, because $F_{i+\tfrac{1}{2}} = F_{(i+1)-\tfrac{1}{2}}$
\end{frame}






\begin{frame}
	\frametitle{Solution Scheme: Problems}
	
	\begin{block}{Direction of information}
		In this problem, all characteristics propagate downstream: Information strictly travels in the flow direction.\\
		$\Rightarrow$ $\rho_i$ mustn't depend on downstream value. Otherwise, you'll be using ``information that isn't there yet''.\\
		$\Rightarrow$ The discretisation will depend on the sign of the velocity $u$.
	\end{block}

	\begin{block}{CFL Condition}
		For explicit integration methods, we need to limit the time step such that information can travel at most by 1 cell during each time step to avoid instabilities.\\
		In 1D: $\Delta t \leq \Delta x / u\quad$ ;  In $N$D: $\Delta t \leq [\sum_i^N u_i / \Delta x_i ]^{-1}$
	\end{block}

	\begin{block}{Numerical Diffusion}
		The solution is not advected perfectly (except for $\Delta t = \frac{\Delta x}{u}$) but smoothed out. The numerical algorithm has a diffusion term as a by-product For a first order algorithm, we're effectively solving $\frac{\partial \rho}{\partial t} + u \frac{\partial \rho}{\partial x} = \frac{\Delta x u}{2}\frac{\partial^2 \rho}{\partial x^2}$
	\end{block}

\end{frame}






\begin{frame}
	\frametitle{Periodic Boundary Conditions}
	
	Simulate a periodic boundary by introducing ghost cells:



	\begin{tikzpicture}[xscale=.9,yscale=1,samples=400, transform shape,every node/.style={scale=.8}]
		%===============
		%	Sequential
		%===============
		
		
		%real Grid
		\filldraw[fill=green!10,draw=teal!80, line width=.15mm] (0, 0) rectangle (8, 1);
		\draw[step=1cm,lightgray,very thin] (0, 0) grid (8,1);
		\draw[draw=teal!80, line width=.15mm] (0, 0) rectangle (8, 1);

		% pattern for real cells
		\fill[pattern=horizontal lines, pattern color=teal!40]  (0,0) rectangle (2, 1);	
		\fill[pattern=horizontal lines, pattern color=teal!40]  (6,0) rectangle (8, 1);	
			
		% Ghost cells left	
		\filldraw[fill=red!20,draw=red!80, line width=.15mm] (-2, 0) rectangle (0, 1);
		\draw[step=1cm,lightgray,very thin] (-2, 0) grid (0,1);
		\draw[draw=red!80, line width=.15mm] (-2, 0) rectangle (0, 1);
		
		%Ghost cells right
		\filldraw[fill=red!20,draw=red!80, line width=.15mm] (8, 0) rectangle (10, 1);
		\draw[step=1cm,lightgray,very thin] (8, 0) grid (10,1);
		\draw[draw=red!80, line width=.15mm] (8, 0) rectangle (10, 1);	


		% Arrows
		\draw[<-, black!70, thick] (-1,1.2) to[bend right=-10] (7,1.2);
		\draw[<-, black!70, thick] (9,-0.2) to[bend right=-10] (1, -0.2);
	
	\end{tikzpicture}
\end{frame}





\section{Piecewise Constant Method}

\begin{frame}
	\frametitle{Piecewise Constant Method: Donor-Cell Advection}
	
	\begin{columns}
		\column{0.6\textwidth}
			Assume cell state within cell is constant.
			
			For $u = 1$:
			\begin{align*}
			\rho_i ^{n+1} &= \rho_i^{n} + 
			\frac{\Delta t}{\Delta x} (f_{i-\half}^{n+\half} - f_{i+\half}^{n+\half} )\\
			f_{i\pm\half}^{n+1} &= u_{i\pm\half}\quad \rho_{i - \half \pm \half}
			\end{align*}
			
			First order accurate in time and space.
			
		\column{.4\textwidth}
			\centering
			\includegraphics[width=\textwidth]{images/donorcell.png}
			\tiny{
				Image adapted from ``Lecture Numerical Fluid Dynamics'', Lecture given by C.P. Dullemond and H.H. Wang at Heidelberg University, 2009
			}
		
	\end{columns}
\end{frame}	






\begin{frame}
	\vspace{10pt}
	\begin{columns}
		\column{.33\textwidth}
			\centering
			\includegraphics[height=.33\textheight]{../results/1D/pwconst/nx=100/plot_advection_step_function_pwconst_nx=100.png}\\
			\includegraphics[height=.33\textheight]{../results/1D/pwconst/nx=1000/plot_advection_step_function_pwconst_nx=1000.png}\\
			\includegraphics[height=.33\textheight]{../results/1D/pwconst/nx=10000/plot_advection_step_function_pwconst_nx=10000.png}
		\column{.33\textwidth}
			\centering
			\includegraphics[height=.33\textheight]{../results/1D/pwconst/nx=100/plot_advection_linear_step_pwconst_nx=100.png}\\
			\includegraphics[height=.33\textheight]{../results/1D/pwconst/nx=1000/plot_advection_linear_step_pwconst_nx=1000.png}\\
			\includegraphics[height=.33\textheight]{../results/1D/pwconst/nx=10000/plot_advection_linear_step_pwconst_nx=10000.png}
		\column{.33\textwidth}
			\centering
			\includegraphics[height=.33\textheight]{../results/1D/pwconst/nx=100/plot_advection_gauss_pwconst_nx=100.png}\\
			\includegraphics[height=.33\textheight]{../results/1D/pwconst/nx=1000/plot_advection_gauss_pwconst_nx=1000.png}\\
			\includegraphics[height=.33\textheight]{../results/1D/pwconst/nx=10000/plot_advection_gauss_pwconst_nx=10000.png}
	\end{columns}
\end{frame}




\section{Piecewise Linear Method}

\begin{frame}
	\frametitle{Piecewise Linear Method}

		Assume that the state in the cell is piecewise linear with some slope $s$. This gives a second-order accurate method.
		\begin{align*}
			\text{For } && x_{i-\half} < x_i < x_{i+\half}: &
			\quad \rho(x, t=t_n) = \rho_i^n + s_i^n(x - x_i)%\\[.5em]			
%			\text{For }&& t_n < t < t_{n+1}: &\quad \rho(x, t) = \rho_i^n +s_i^n(x -[x_i + u (t - t_n)])
		\end{align*}
%		
		The flux over the interface is then
%
		\begin{align*}
			f_{i-\half}(t) 	&= u \rho(x = x_{i-\half}, t)\\
						&= u \rho_{i-1} +  u s_i^n(\nicefrac{\Delta x}{2} - u(t-t_n))
		\end{align*}
%
		Finally averaging the fluxes over a time step gives:
		\begin{align*}
			\rho_i^{n+1} = \rho_i^n - \frac{u \Delta t}{\Delta x} ( \rho_i ^n - \rho_{i-1}^n) - \frac{u \Delta t}{\Delta x} \frac{1}{2} (s_i^n - s_{i-1}^n)(\Delta x - u\Delta t)
		\end{align*}
		
		Choice of slope: $s_i^n = \frac{\rho_{i+1}^n - \rho_i^n}{\Delta x}$ ( Lax-Wendroff method )

\end{frame}









\begin{frame}
	\vspace{10pt}
	\begin{columns}
		\column{.33\textwidth}
			\centering
			\includegraphics[height=.33\textheight]{../results/1D/pwlin/nx=100/plot_advection_step_function_pwlin_nx=100.png}\\
			\includegraphics[height=.33\textheight]{../results/1D/pwlin/nx=1000/plot_advection_step_function_pwlin_nx=1000.png}\\
			\includegraphics[height=.33\textheight]{../results/1D/pwlin/nx=10000/plot_advection_step_function_pwlin_nx=10000.png}
		\column{.33\textwidth}
			\centering
			\includegraphics[height=.33\textheight]{../results/1D/pwlin/nx=100/plot_advection_linear_step_pwlin_nx=100.png}\\
			\includegraphics[height=.33\textheight]{../results/1D/pwlin/nx=1000/plot_advection_linear_step_pwlin_nx=1000.png}\\
			\includegraphics[height=.33\textheight]{../results/1D/pwlin/nx=10000/plot_advection_linear_step_pwlin_nx=10000.png}
		\column{.33\textwidth}
			\centering
			\includegraphics[height=.33\textheight]{../results/1D/pwlin/nx=100/plot_advection_gauss_pwlin_nx=100.png}\\
			\includegraphics[height=.33\textheight]{../results/1D/pwlin/nx=1000/plot_advection_gauss_pwlin_nx=1000.png}\\
			\includegraphics[height=.33\textheight]{../results/1D/pwlin/nx=10000/plot_advection_gauss_pwlin_nx=10000.png}
	\end{columns}
\end{frame}





\begin{frame}
	\begin{block}{New Problem: Oscillations}
		The piecewise linear elements can have overshoots, leading to the oscillations seen in the previous plots.

		
		\begin{center} 
			\includegraphics[height=0.2\textheight]{images/overshoot.png}
			
			\tiny{
				Image adapted from ``Lecture Numerical Fluid Dynamics'', Lecture given by C.P. Dullemond and H.H. Wang at Heidelberg University, 2009
			}
		\end{center}
			
		Godunov's theorem: \textit{any linear algorithm for
			solving partial differential equations, with the property of not producing new extrema, can be at
			most first order.}
		
		$\Rightarrow$ use non-linear conditions (slope limiters) to modify the slope $s_i^n$ to prevent overshoots. Requirement: total variation diminishing: $TV(\rho^{n+1}) \leq TV(\rho^n) \equiv \sum |\rho_i - \rho_{i-1}|$. Such a scheme will not develop oscillations near a jump, because a jump is a monotonically in/decreasing function and a TVD scheme will not increase the $TV$.
	\end{block}
\end{frame}



\section{Minmod Slope Limiter}

\begin{frame}
	\frametitle{Minmod Slope Limiter}
	
	\begin{align*}
		s_i^n = \mathrm{minmod} \left( \frac{\rho_i^n - \rho_{i-1}^n}{\Delta x}, \frac{\rho_{i+1}^n - \rho_i^n}{\Delta x} \right)
	\end{align*}
	with
	\begin{align*}
		\mathrm{minmod}(a, b) = \begin{cases}
		 	a & \text{ if } |a| < |b| \text{ and } ab > 0\\		
		 	b & \text{ if } |a| > |b| \text{ and } ab > 0\\
			0 & \text{ if } ab < 0
								\end{cases}
	\end{align*}
\end{frame}






\begin{frame}
	\vspace{10pt}
	\begin{columns}
		\column{.33\textwidth}
			\centering
			\includegraphics[height=.33\textheight]{../results/1D/minmod/nx=100/plot_advection_step_function_minmod_nx=100.png}\\
			\includegraphics[height=.33\textheight]{../results/1D/minmod/nx=1000/plot_advection_step_function_minmod_nx=1000.png}\\
			\includegraphics[height=.33\textheight]{../results/1D/minmod/nx=10000/plot_advection_step_function_minmod_nx=10000.png}
		\column{.33\textwidth}
			\centering
			\includegraphics[height=.33\textheight]{../results/1D/minmod/nx=100/plot_advection_linear_step_minmod_nx=100.png}\\
			\includegraphics[height=.33\textheight]{../results/1D/minmod/nx=1000/plot_advection_linear_step_minmod_nx=1000.png}\\
			\includegraphics[height=.33\textheight]{../results/1D/minmod/nx=10000/plot_advection_linear_step_minmod_nx=10000.png}
		\column{.33\textwidth}
			\centering
			\includegraphics[height=.33\textheight]{../results/1D/minmod/nx=100/plot_advection_gauss_minmod_nx=100.png}\\
			\includegraphics[height=.33\textheight]{../results/1D/minmod/nx=1000/plot_advection_gauss_minmod_nx=1000.png}\\
			\includegraphics[height=.33\textheight]{../results/1D/minmod/nx=10000/plot_advection_gauss_minmod_nx=10000.png}
	\end{columns}
\end{frame}



\section{Van Leer Slope Limiter}

\begin{frame}
	\frametitle{Van Leer Slope Limiter}
	Rewrite flux (assuming $u = 1$) as
	\begin{align*}
		f_{i-\half}^{n+\half} &= u \rho_{i-1} + \frac{1}{2} u \left( 1 - \frac{u\Delta t}{\Delta x} \right) \phi(r_{i-\half}^n) (q_i^n - q_{i-1}^n)\\
		r_{i-\half}^n &= \frac{q_{i-1}^n - q_{i-2}^n}{q_i^n - q_{i-1}^n}
	\end{align*}
	Here, $\phi$ is the slope/flux limiter.
	
	The Van Leer flux limiter is defined as
	\begin{align*}
		\phi(r) = \frac{r - |r|}{1 - |r|}
	\end{align*}
\end{frame}







\begin{frame}
	\vspace{10pt}
	\begin{columns}
		\column{.33\textwidth}
			\centering
			\includegraphics[height=.33\textheight]{../results/1D/VanLeer/nx=100/plot_advection_step_function_VanLeer_nx=100.png}\\
			\includegraphics[height=.33\textheight]{../results/1D/VanLeer/nx=1000/plot_advection_step_function_VanLeer_nx=1000.png}\\
			\includegraphics[height=.33\textheight]{../results/1D/VanLeer/nx=10000/plot_advection_step_function_VanLeer_nx=10000.png}
		\column{.33\textwidth}
			\centering
			\includegraphics[height=.33\textheight]{../results/1D/VanLeer/nx=100/plot_advection_linear_step_VanLeer_nx=100.png}\\
			\includegraphics[height=.33\textheight]{../results/1D/VanLeer/nx=1000/plot_advection_linear_step_VanLeer_nx=1000.png}\\
			\includegraphics[height=.33\textheight]{../results/1D/VanLeer/nx=10000/plot_advection_linear_step_VanLeer_nx=10000.png}
		\column{.33\textwidth}
			\centering
			\includegraphics[height=.33\textheight]{../results/1D/VanLeer/nx=100/plot_advection_gauss_VanLeer_nx=100.png}\\
			\includegraphics[height=.33\textheight]{../results/1D/VanLeer/nx=1000/plot_advection_gauss_VanLeer_nx=1000.png}\\
			\includegraphics[height=.33\textheight]{../results/1D/VanLeer/nx=10000/plot_advection_gauss_VanLeer_nx=10000.png}
	\end{columns}
\end{frame}






\section{Precision of the Algorithms}
\begin{frame}
	\frametitle{Precision of the Algorithms}
	Quantify error through $L1$ error norm: $L1 = \frac{1}{N} \sum\limits_i |\rho_i - \tilde{\rho}(x_i)|$, where $\tilde{\rho}(x_i)$ is the analytical solution.\\[2em]
	
	For $t = 100$:\\\vfill
	
	\begin{columns}
		\column{.37\textwidth}
			\includegraphics[width=\textwidth]{../results/1D/errors/errorplot_advection_step_function_t=100.0.png}
		\column{.37\textwidth}
			\includegraphics[width=\textwidth]{../results/1D/errors/errorplot_advection_linear_step_t=100.0.png}
		\column{.37\textwidth}	
			\includegraphics[width=\textwidth]{../results/1D/errors/errorplot_advection_gauss_t=100.0.png}	
	\end{columns}
	
\end{frame}






\section{2D Advection}
\include{7-2d-advection}
\include{7-2d-x}
\include{7-2d-y}
\include{7-2d-diagonal}






\begin{frame}[fragile]
	Program, plotting scripts and this presentation available on \verb!https://bitbucket.org/mivkov/computational_astrophysics!
\end{frame}













\end{document}
