\documentclass[10pt]{beamer}


\usepackage{lmodern} 		% Diese beiden packages sorgen für echte 
\usepackage[T1]{fontenc}	% Umlaute.

\usepackage{amssymb, amsmath, color, graphicx, float, setspace, tipa}
\usepackage[utf8]{inputenc} 
\usepackage[english]{babel}
\usepackage[justification=centering]{caption}
\addto\captionsenglish{\renewcommand{\figurename}{}} %Abbildungen nicht bzw. anders beschriften.


%\usepackage[pdfpagelabels,pdfstartview = FitH,bookmarksopen = true,bookmarksnumbered = true,linkcolor = black,plainpages = false,hypertexnames = false,citecolor = black, breaklinks]{hyperref}
%\usepackage{url}
%\usepackage{picins} 		%Gleittext um Grafik. Befehl: parpic. Vorlage siehe unten
\usepackage{longtable} 		%Seitenübergreifende Tabelle. Vorlage siehe unten
\newtheorem*{bem}{Bemerkung} % Neue Theorem-Umgebung: Bemerkung
\newcommand{\fillframe}{\vskip0pt plus 1filll} 
\newcommand{\musr}{$\mu$SR }

\usepackage{units}
\newcommand{\half}{\nicefrac{1}{2}}

\usepackage{tikz}
\usetikzlibrary{patterns}

\usepackage{grffile} %allow image filenames.that.include.dots.png



%-----------------
%BEAMER-SPEZIFISCH
%-----------------


\usetheme{metropolis}
%deactivate a new page when a new section begins
\metroset{sectionpage=none} 
\usepackage{FiraSans}
\usefonttheme[onlymath]{serif}



% Verschiedene Varianten von usetheme, usecolortheme und usefonttheme kann man hier ausprobieren: http://deic.uab.es/~iblanes/beamer_gallery/


%Halbtransparente Overlays (was als nächstes Element auf der Folie gezeigt wird)
%\setbeamercovered{transparent} 

% Entfernt Navigationssymbole unten
%\beamertemplatenavigationsymbolsempty 

% Seitenzahlen als links
%\setbeamertemplate{footline}[frame]  
%    \setbeamertemplate{footline}{%
%    	\raisebox{5pt}{\makebox[\paperwidth]{\hfill\makebox[10pt]{\hyperlink{tableofcontents}{\scriptsize\insertframenumber}}}}}

